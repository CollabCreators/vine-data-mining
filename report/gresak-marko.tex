% To je predloga za poročila o domačih nalogah pri predmetih, katerih
% nosilec je Tomaž Curk. Avtor predloge je Blaž Zupan.
%
% Seveda lahko tudi dodaš kakšen nov, zanimiv in uporaben element,
% ki ga v tej predlogi (še) ni. Več o LaTeX-u izveš na
% spletu, na primer na http://tobi.oetiker.ch/lshort/lshort.pdf.
%
% To predlogo lahko spremeniš v PDF dokument s pomočjo programa
% pdflatex, ki je del standardne instalacije LaTeX programov.

\documentclass[a4paper,11pt]{article}
\usepackage{a4wide}
\usepackage{fullpage}
\usepackage[utf8x]{inputenc}
\usepackage[slovene]{babel}
\selectlanguage{slovene}
\usepackage[toc,page]{appendix}
\usepackage[pdftex]{graphicx} % za slike
\usepackage{setspace}
\usepackage{multicol}
\usepackage{color}
\definecolor{light-gray}{gray}{0.95}
\usepackage{listings} % za vključevanje kode
\usepackage{minted} % za barvanje kode
\usepackage{hyperref}
\renewcommand{\baselinestretch}{1.2} % za boljšo berljivost večji razmak
\renewcommand{\appendixpagename}{Priloge}


\newminted{ts}{
  % remove first 2 characters (padding spaces)
  gobble = 2,
  % set left line (code block padding hack)
  frame = leftline,
  % 0 width for frame ruler (gives padding but is invisible)
  framerule = 0pt,
  % distance between frame and content
  framesep = 8pt,
  % set background
  bgcolor = light-gray
}
% usage:
% \begin{tscode}
%   // Some great code.
% \end{tscode}

\title{\huge{Končno poročilo} \\ \Large{\em{Mreže v socialnem omrežju Vine}}}
\author{Marko Grešak (63130058)}
\date{\today}

\begin{document}

\maketitle

\section{Uvod}

V svoji projektni nalogi sem analiziral podatke socialnega omrežja \href{https://vine.co}{\underline{Vine}}. Pri tem je bilo eno bolj zanimivh področij problem, kako kar se da dobro prikazati sodelovanje uporabnikov. Poskušal sem tudi odkriti, kakšne skupnosti se tvorijo čez čas, vendar mi to zaradi pomanjkana podatkov ni uspelo. Pred začetkom idelave naloge sem po raziskavi ugotovil, da še nihče ni delal podobne raziskave oziroma je ni objavil javno.

\subsection{Kaj je Vine?}

Ko govorim o projektu se v večini primerov najprej pojavi vprašanje: \textit{“Kaj je Vine?”}. Vine je Twitterjeva spletna storitev, ki uporabnikom omogoča nalaganje do 7 sekundnih videov, s časom pa so ga uporabniki začeli uporabljati kot čisto neodvisno socialno omrežje. Omrežje je postalo popularno zaradi uporabnikov, ki ustvarjajo smešno vsebino.

\subsection{Podatki in Tehnologije}

\begin{itemize}
  \item Omejil sem se na uporabnike z vsaj 50.000 sledilci, teh sem našel 588,
  \item Za razvojno okolje sem si izbral \href{https://nodejs.org/}{\underline{Node.js}} in \href{http://www.typescriptlang.org/}{\underline{TypeScript}},
  \item Večino grafov sem narial iz Node.js s pomočjo storitve \href{https://plot.ly/}{\underline{Plotly}}, z izjemo grafa sodelovanja med uporabniki, ki je narisan s pomočjo \href{http://d3js.org/}{\underline{D3.js}}.
\end{itemize}

\subsection{Programska koda}
Programska koda je javno dostopna na GitHubu, na naslovu\\
\href{https://github.com/markogresak/vine-data-mining}{\underline{https://github.com/markogresak/vine-data-mining}}

\subsection{Pojmi}
\begin{itemize}
  \item API - \href{http://en.wikipedia.org/wiki/Application_programming_interface}{\underline{Application programming interface}},
  \item \href{http://sl.wikipedia.org/wiki/Predpone_SI}{\underline{k, M, B}} - 1.000 (tisoč), 1.000.000 (milijon), 1.000.000.000 (milijarda), v tem zaporedju.
\end{itemize}

\pagebreak
\section{Podatki}

\subsection{Izvor podatkov}

Med iskanem podatkov nisem našel nobene povezave, z že vnaprej pripravljenimi podatki. To pomeni, da vse svoje podatke črpam zgolj iz \href{https://api.vineapp.com}{\underline{Vine API}}. Ker API nima nikakršne uradne dokumentacije, sem si moral pomagati redkimi, predvsem pa zastarelimi, viri na spletu ter svojim raziskovenjam, iz tega pa sem izpeljal dokumentacijo \href{https://github.com/markogresak/vine-data-mining/blob/master/API-reference.md}{\underline{Vine API Reference}}. Že ob času pisanja tega poročila so nekatere informacije zastarele, ampak se jih nisem spravil popravljati, saj sem bil prezaposljen s popravljanjem same kode.
\par
Na tej točki sta se že pojavila prva dva problema: pomanjkanje dokumentacije ter večkratno spreminjanje API-ja, zaradi česar je vsa najdena dokumentcija hitro zastarela. Samo v času izdelave tega projekta se je po mojem opažanju odgovor API-ja spremenil vsaj trikrat. Sicer spremembe niso bile drastične, ampak so bile dovolj, da sem moral svoj program popravljati, da je lahko nadaljeval, pojavili pa so se tudi nekonsistentni podatki za iste odgovore APIja, zato sem se po drugi spremembi odločil, da ne shranjujem direktnih odgovorov, ampak svoje, prilagojene vrednosti.

\end{document}

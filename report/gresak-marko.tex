% To je predloga za poročila o domačih nalogah pri predmetih, katerih
% nosilec je Tomaž Curk. Avtor predloge je Blaž Zupan.
%
% Seveda lahko tudi dodaš kakšen nov, zanimiv in uporaben element,
% ki ga v tej predlogi (še) ni. Več o LaTeX-u izveš na
% spletu, na primer na http://tobi.oetiker.ch/lshort/lshort.pdf.
%
% To predlogo lahko spremeniš v PDF dokument s pomočjo programa
% pdflatex, ki je del standardne instalacije LaTeX programov.

\documentclass[a4paper,11pt]{article}
\usepackage{a4wide}
\usepackage{fullpage}
\usepackage[utf8x]{inputenc}
\usepackage[slovene]{babel}
\selectlanguage{slovene}
\usepackage[toc,page]{appendix}
\usepackage[pdftex]{graphicx} % za slike
\usepackage{setspace}
\usepackage{multicol}
\usepackage{color}
\definecolor{light-gray}{gray}{0.95}
\usepackage{listings} % za vključevanje kode
\usepackage{minted} % za barvanje kode
\usepackage{hyperref}
\renewcommand{\baselinestretch}{1.2} % za boljšo berljivost večji razmak
\renewcommand{\appendixpagename}{Priloge}


\newminted{ts}{
  % remove first 2 characters (padding spaces)
  gobble = 2,
  % set left line (code block padding hack)
  frame = leftline,
  % 0 width for frame ruler (gives padding but is invisible)
  framerule = 0pt,
  % distance between frame and content
  framesep = 8pt,
  % set background
  bgcolor = light-gray
}
% usage:
% \begin{tscode}
%   // Some great code.
% \end{tscode}

\title{\huge{Končno poročilo} \\ \Large{\em{Mreže v socialnem omrežju Vine}}}
\author{Marko Grešak (63130058)}
\date{\today}

\begin{document}

\maketitle

\section{Uvod}

V svoji projektni nalogi sem analiziral podatke socialnega omrežja \href{https://vine.co}{\underline{Vine}}. Pri tem je bilo eno bolj zanimivh področij problem, kako kar se da dobro prikazati sodelovanje uporabnikov. Poskušal sem tudi odkriti, kakšne skupnosti se tvorijo čez čas, vendar mi to zaradi pomanjkana podatkov ni uspelo. Pred začetkom idelave naloge sem po raziskavi ugotovil, da še nihče ni delal podobne raziskave oziroma je ni objavil javno.

\subsection{Kaj je Vine?}

Ko govorim o projektu se v večini primerov najprej pojavi vprašanje: \textit{“Kaj je Vine?”}. Vine je Twitterjeva spletna storitev, ki uporabnikom omogoča nalaganje do 7 sekundnih videov, s časom pa so ga uporabniki začeli uporabljati kot čisto neodvisno socialno omrežje. Omrežje je postalo popularno zaradi uporabnikov, ki ustvarjajo smešno vsebino.

\subsection{Podatki in Tehnologije}

\begin{itemize}
  \item Omejil sem se na uporabnike z vsaj 50.000 sledilci, teh sem našel 588,
  \item Za razvojno okolje sem si izbral \href{https://nodejs.org/}{\underline{Node.js}} in \href{http://www.typescriptlang.org/}{\underline{TypeScript}},
  \item Večino grafov sem narial iz Node.js s pomočjo storitve \href{https://plot.ly/}{\underline{Plotly}}, z izjemo grafa sodelovanja med uporabniki, ki je narisan s pomočjo \href{http://d3js.org/}{\underline{D3.js}}.
\end{itemize}

\subsection{Programska koda}
Programska koda je javno dostopna na GitHubu, na naslovu\\
\href{https://github.com/markogresak/vine-data-mining}{\underline{https://github.com/markogresak/vine-data-mining}}

\subsection{Pojmi}
\begin{itemize}
  \item API - \href{http://en.wikipedia.org/wiki/Application_programming_interface}{\underline{Application programming interface}},
  \item \href{http://sl.wikipedia.org/wiki/Predpone_SI}{\underline{k, M, B}} - 1.000 (tisoč), 1.000.000 (milijon), 1.000.000.000 (milijarda), v tem zaporedju.
\end{itemize}

\pagebreak
\section{Podatki}

\subsection{Izvor podatkov}

Med iskanem podatkov nisem našel nobene povezave, z že vnaprej pripravljenimi podatki. To pomeni, da vse svoje podatke črpam zgolj iz \href{https://api.vineapp.com}{\underline{Vine API}}. Ker API nima nikakršne uradne dokumentacije, sem si moral pomagati redkimi, predvsem pa zastarelimi, viri na spletu ter svojim raziskovenjam, iz tega pa sem izpeljal dokumentacijo \href{https://github.com/markogresak/vine-data-mining/blob/master/API-reference.md}{\underline{Vine API Reference}}. Že ob času pisanja tega poročila so nekatere informacije zastarele, ampak se jih nisem spravil popravljati, saj sem bil prezaposljen s popravljanjem same kode.
\\
Na tej točki sta se že pojavila prva dva problema: pomanjkanje dokumentacije ter večkratno spreminjanje API-ja, zaradi česar je vsa najdena dokumentcija hitro zastarela. Samo v času izdelave tega projekta se je po mojem opažanju odgovor API-ja spremenil vsaj trikrat. Sicer spremembe niso bile drastične, ampak so bile dovolj, da sem moral svoj program popravljati, da je lahko nadaljeval, pojavili pa so se tudi nekonsistentni podatki za iste odgovore APIja, zato sem se po drugi spremembi odločil, da ne shranjujem direktnih odgovorov, ampak svoje, prilagojene vrednosti.
\\
Naslednji problem se je pojavil pri ID vrednostih, saj so vse večje od limita vrednosti za številko v JavaScriptu oziroma TypeScript, jezik v katerem sem napisal mojo nalogo. Razlaga za to je, da je največja cela številčna vrednost v JavaScriptu \(2^{52}\), kar pa je manj od ID vrednosti, katera je bila npr. \textit{934940633704046592}, zato se je zgodil preliv (ang. overflow), na koncu pa je bila vrednost iz primera enaka \textit{934940633704046600}. To seveda pomeni, da uporabnika po tej vrednosti ne morem najti, saj je shranjena vrednost različna od pravega idja entitete.
\\
Problem sem rešil tako, da sem začel pri 5 po mojem mnenju najbolj sledenih uporabnikih, pri katerih je verjetnost sodelovanja večja, saj imajo veliko število objav. Za vsakega od teh uporabnikov sem pripravil zahtevo na profil ter zahtevo na časovnico, kjer so vsi posnetki. Tukaj se pojavi omejitev, saj lahko v zahtevi dobimo največ 100 posnetkov, za več pa moramo zahtevati naslednje strani. Ostale

\begin{minted}{json}

\end{minted}

\section{Metode}

Tu opišeš, na kakšen način si rešil nalogo (tehnike in metode, ki si
jih uporabil). Lahko vključiš tudi zanimiv del programske kode, ki
si jo morda pri tem razvil ali pa v poročilo dodatno vključiš sliko,
kot je na primer slika~\ref{slika1}. Vse slike in tabele, ki jih
vključiš v poročilo, morajo biti navedene v besedilu oziroma se moraš
na njih sklicati.

\begin{figure}[htbp]
\begin{center}
\includegraphics[scale=0.3]{slika-primer.png}
\caption{Vsako sliko opremi s podnapisom, ki pove, kaj slika prikazuje.}
\label{slika1}
\end{center}
\end{figure}

V to poglavje lahko tudi vključiš kakšen metodološko zanimiv del
kode. Primer vključitve kode oziroma implementirane funkcije v
programskem jeziku Python je:

\begin{lstlisting}
def fib (n):
    if n == 0:
        return 0
    elif n == 1:
        return 1
    else:
        return fib (n-1) + fib (n-2)
\end{lstlisting}

Izris te kode je lahko sicer tudi lepši, poskušaš lahko najti še
primernejši način vključevanja kode v Pythonu oziroma v tvojem izbranem
programskem jeziku v okolje \LaTeX{}.

\section{Rezultati}

V tem poglavju podaš rezultate s kratkim (enoodstavčnim)
komentarjem. Rezultate lahko prikažeš tudi v tabeli (primer je
tabela~\ref{tab1}).

Odstavke pri pisanju poročila v LaTeX-u ločiš tako, da pred novim
odstavkom pustiš prazno vrstico. Tudi, če pišeš poročilo v kakšnem
drugem urejevalniku, morajo odstavki biti vidno ločeni. To narediš z
zamikanjem ali pa z dodatnim presledkom.

\begin{table}[htbp]
\caption{Atributi in njihove zaloge vrednosti.}
\label{tab1}
\begin{center}
\begin{tabular}{llp{3cm}}
\hline
ime spremenljivke & definicijsko območje & opis \\
\hline
cena & [0, 500] & cena izdelka v EUR\\
teža & [1, 1000] & teža izdelka v dag \\
kakovost & [slaba|srednja|dobra] & kakovost izdelka \\
\hline
\end{tabular}
\end{center}
\end{table}

Podajanje rezultati naj bo primerno strukturirano. Če ima naloga več
podnalog, uporabi podpoglavja. Če bi želel poročati o rezultatih
izčrpno in pri tem uporabiti vrsto tabel ali grafov, razmisli o
varianti, kjer v tem poglavju prikažeš in komentiraš samo glavne
rezultate, kakšne manj zanimive detajle pa vključite v prilogo (glej
prilogi~\ref{app-res} in~\ref{app-code}).

\section{Izjava o izdelavi domače naloge}
Domačo nalogo in pripadajoče programe sem izdelal sam.

\appendix
\appendixpage
\section{\label{app-res}Podrobni rezultati poskusov}

Če je rezultatov v smislu tabel ali pa grafov v nalogi mnogo,
predstavi v osnovnem besedilu samo glavne, podroben prikaz
rezultatov pa lahko predstaviš v prilogi. V glavnem besedilu ne
pozabi navesti, da so podrobni rezultati podani v prilogi.

\section{\label{app-code}Programska koda}

Za domače naloge bo tipično potrebno kaj sprogramirati. Če ne bo od
vas zahtevano, da kodo oddate posebej, to vključite v prilogo. Čisto
za okus sem tu postavil nekaj kode, ki uporablja Orange
(\url{http://www.biolab.si/orange}) in razvrščanje v skupine.


\begin{lstlisting}
import random
import Orange

data_names = ["iris", "housing", "vehicle"]
data_sets = [Orange.data.Table(name) for name in data_names]

print "%10s %3s %3s %3s" % ("", "Rnd", "Div", "HC")
for data, name in zip(data_sets, data_names):
    random.seed(42)
    km_random = Orange.clustering.kmeans.Clustering(data, centroids = 3)
    km_diversity = Orange.clustering.kmeans.Clustering(data, centroids = 3,
        initialization=Orange.clustering.kmeans.init_diversity)
    km_hc = Orange.clustering.kmeans.Clustering(data, centroids = 3,
        initialization=Orange.clustering.kmeans.init_hclustering(n=100))
    print "%10s %3d %3d %3d" % (name, km_random.iteration, \
    km_diversity.iteration, km_hc.iteration)
\end{lstlisting}

\end{document}
